\date{\today}

\documentclass[12pt]{article}
\usepackage{titlesec}
\usepackage[a4paper, margin=1in]{geometry}
\titleformat{\section}
  {\normalfont\fontsize{12}{15}\bfseries}{\thesection}{1em}{}

\begin{document}
\begin{center}
\textbf{\large{Part 2 Dissertation Progress Report\\}}
\vspace{3mm}
\textbf{\large{Learning the Stock Market: Deep Learning and Sentiment Analysis-Based Stock Price Prediction\\}}
\vspace{3mm}
\normalsize{Ian Tai \\
		ihbt2@cam.ac.uk\\}
\end{center}
Supervisors: Dr. Sean Holden, Prof. Stephen Satchell\\
Director of Studies: Dr. Sean Holden\\
Overseers: Prof. Anuj Dawar, Dr. Timothy Griffin\\

\section*{Summary of Work Completed}

\paragraph{Outline}
\begin{enumerate}
\item Finished data collection for financial history data
\item Built LSTM system for price prediction
\item Conducted large-scale data collection for news headline
data via web-scraping
\item Built Naive Bayes classifier for sentiment classification
\item Implemented semi-supervised learning for sentiment classification
using K-Nearest-Neighbours and Support Vector Machine with radial 
basis function kernel
\item Built data processing system for end-to-end data collection to
model training integration
\end{enumerate}

\paragraph{Financial Data:}
Collected historical stock data of companies in the NASDAQ 100 index
with a time interval of 5 minutes from various Bloomberg terminals 
around the University. This data included opening price, volume 
traded, bid price, bid volume, ask price, and ask volume.

\paragraph{LSTM:}
Built deep LSTM system which supports multi-layer configurations. The
model currently correctly predicts the rise and fall of a single stock 
at around 52\% accuracy after repeated training with normally-randomized 
initial weights. Since the system currently only utilizes historical
prices from the same single stock, this accuracy is acceptable and
should increase when our features become more sophisticated.

\paragraph{News Data:}
Experimented with different Twitter API libraries before finally settling
for a web-scraper instead because of numerous advantages such as speed
and usability. Collected around 1.4 million tweets from the 35 biggest
news headlines in the world, spanning the past year. I randomly sampled
a small subset (300 tweets) to manually classify sentiment.

\paragraph{Naive Bayes:}
Implemented a Naive Bayes classifier as the first steps towards
a viable sentiment classification model. Tested on a small subset
of the large twitter dataset, using feature generation techniques
such as Bag of Words and Doc2Vec. The results were unsatisfactory,
which led to the development of more sophisticated models.

\paragraph{Semi-Supervised Learning:}
Used popular Machine Learning library scikit-learn to implement
semi-supervised learning for a larger subset of the twitter data.
Results were also unsatisfactory, which led me to believe that dataset
may be too noisy, or the classification approach incorrect. I will spend
more time on this portion in the near future.

\paragraph{Data Processing:}
Built a simple data processing system that wrote inputs and outputs to
temporary files to convert to the corresponding format for different
libraries and self-built modules.\\

\section*{Difficulties}
There has only been one main unexpected difficulty:\\

\noindent None of the aforementioned sentiment classifiers performed to a
sufficient standard, with the best result (RBF kernel of SVM) having 
only a 57\% accuracy on the validation set, after much experimentation
with different classifiers and knowledge representations.\\

\noindent I will need to fix this by either changing my dataset, 
or use a specific subset that doesn't include noisy data, 
or change my representation of the input data within the model.\\


\section*{Schedule}
Because of the difficulty outlined above, the sentiment analysis
portion of my project, which was supposed to be completed before
Christmas, is yet to be fully completed. However, the rest of the
project is on track. This setback puts me around 2 weeks behind.\\

\noindent I will try to fix this by allocating time to the sentiment analysis 
problem in the next stage of work, which initially included only 
dissertation writing.

\end{document}